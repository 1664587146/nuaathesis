% use \nuaaset to provide some basic information (in Chinese)
\nuaaset{
  title = {\nuaathesis{} 英文论文示例},
  author = {佚名},
  college = {\TeX{} 学院},
  advisers = {Knuth\quad 教授},
  % applydate = {二〇一六年三月}  % default to current date
  %
  % bachelor only
  major = {\LaTeX{} 科学与技术},
  studentid = {131810299},
  classid = {1318001},
  % master/doctor only
  majorsubject = {编程与艺术},
  researchfield = {轮子制造},
  libraryclassid = {H319},        % 中图分类号
  subjectclassid = {050211},      % 学科分类号
  thesisid = {1028712 16-S022},   % 论文编号
}

% use \nuaasetEn to setup the cover (in English)
\nuaasetEn{
  title = {``How to'' Write \textit{Thesis} in English \linebreak with \nuaathesis},
  author = {nuaatug},
  college = {College of \TeX},
  majorsubject = {Programming and Typesetting},
  advisers = {Professor~Knuth},
  degreefull = {Master of Arts},
  % applydate = {March, 8012}
}

% abstract, Chinese
\begin{abstract}
本文主要演示英文论文写作时的注意事项。

大部分中文 \LaTeX{} 的内容同样适用于英文,在此不再赘述。
\end{abstract}
\keywords{英语, 注意事项}

% abstract, English
\begin{abstractEn}
In this document, we will demonstrate how to write thesis with \nuaathesis.

Because both English and Chinese essays use the same document class,
please refer to the Chinese demo for common features.
This document only focuses on English-specific features.
\end{abstractEn}
\keywordsEn{English, thesis writing}


% load packages, define global settings/macros
\newcommand\cs[1]{\texttt{\textbackslash#1}\xspace}
\theoremstyle{nuaaplain}
\nuaatheoremchapu{definition}{Definition}
\nuaatheoremchapu{assumption}{Assumption}

\usepackage{subfig}
\usepackage{rotating}
\usepackage[usenames,dvipsnames]{xcolor}
\usepackage{metalogo}
\usepackage{tikz}
\usepackage{pgfplots}
\usepackage{ifthen}
\usepackage{longtable}
\usepackage{siunitx}
\usepackage{listings}
\usepackage{multirow}
\usepackage{xspace}

\lstdefinestyle{lstStyleBase}{%
  basicstyle=\small\ttfamily,
  aboveskip=\medskipamount,
  belowskip=\medskipamount,
  lineskip=0pt,
  boxpos=c,
  showlines=false,
  extendedchars=true,
  upquote=true,
  tabsize=2,
  showtabs=false,
  showspaces=false,
  showstringspaces=false,
  numbers=left,
  numberstyle=\footnotesize,
  linewidth=\linewidth,
  xleftmargin=\parindent,
  xrightmargin=0pt,
  resetmargins=false,
  breaklines=true,
  breakatwhitespace=false,
  breakindent=0pt,
  breakautoindent=true,
  columns=flexible,
  keepspaces=true,
  framesep=3pt,
  rulesep=2pt,
  framerule=1pt,
  backgroundcolor=\color{gray!5},
  stringstyle=\color{green!40!black!100},
  keywordstyle=\bfseries\color{blue!50!black},
  commentstyle=\slshape\color{black!60}}

\lstdefinestyle{lstStyleShell}{%
    style=lstStyleBase,
    frame=l,
    rulecolor=\color{blue},
    language=bash}

\lstdefinestyle{lstStyleLaTeX}{%
    style=lstStyleBase,
    frame=l,
    rulecolor=\color{cyan},
    language=[LaTeX]TeX}

\lstnewenvironment{latex}{\lstset{style=lstStyleLaTeX}}{}
\lstnewenvironment{shell}{\lstset{style=lstStyleShell}}{}


%\usetikzlibrary{external}
%\tikzexternalize % activate!

