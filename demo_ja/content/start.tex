\chapter{はじめに}

日本語専修じゃないので、中国語で書きたいっと思います。

\zhcn{除了中文、英文外,\nuaathesis{} 也支持日语论文,论文主体为日语,其中偶尔出现小段汉语。

由于目标用户的母语是汉语,所以请读者参阅用汉语写的文档 nuaathesis.pdf。
中文的示例文档中大部分内容是通用的,里面包含一些常用宏包的使用方法。
本文档只介绍一些日语 \LaTeX{} 的使用注意事项。}


\section{\zhcn{\LaTeX{} 环境准备}}

\zhcn{请参照文档、中英文示例论文中介绍的方法,该方法也会一同安装日语 \LaTeX{} 所需要的工具、宏包。}

\section{\zhcn{编译论文}}

\zhcn{这一步与其他语言的论文有很大不同,由于使用了 \CTeX{} 来提供中文支持,
因此只能使用 up\LaTeX{} 引擎来编译。编译前需要确认论文目录下存在以下文件:}
\begin{itemize}
  \item \zhcn{nuaathesis.cls、nuaathesis.bst 等文件(与其他语言相同)}
  \item \zhcn{ctex-engine-uptex.def(请参阅 pdf 文档,里面描述了如何获取、修改该文件)}
  \item \zhcn{.latexmkrc(如果打算使用 latexmk 来编译的话)}
\end{itemize}

\zhcn{编译文档时推荐使用 latexmk,假定论文的名字叫做 master.tex,具体执行的命令如下:}
\begin{shell}
latexmk -pdfdvi master
\end{shell}

\zhcn{实质上,它执行了以下几条命令:}
\begin{shell}
uplatex master
upbibtex master
uplatex master
uplatex master
dvipdfmx master
\end{shell}
