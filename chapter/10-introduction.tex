%# -*- coding: utf-8-unix -*-
\chapter{简介}\label{chap:intro}

这是南京航空航天大学(非官方)学位论文\LaTeX 模板,当前模板的版本是\version。本模板由\nuaathesis~Group共同开发,模板文档由Old Jack和yzwduck撰写。

本模板最早可以追溯到人人网上的一篇博客\footnote{\url{http://blog.renren.com/share/546499630/17959595570}},由黄大宁、邓欣珂、徐添豪、石坤四人共同开发完善,参考了当时东南大学的\seuthesix 模板;除此之外在Github上也可以找到一个repo\footnote{\url{https://github.com/naa803/nuaa-thesis}},由Felix Ding、Jun Wang、Jackie Hou三位老师和Vevi Zhong同学共同维护,但是repo中的.cls和.sty文件是空文件。该repo目前基于已停止更新的旧repo\footnote{\url{https://github.com/JackWzh/nuaathesis}},本模板的参考文献格式相比之更符合南航目前的参考文献标准,其他细节也更出色,故仍推荐使用我们的模板。

回顾人人网的模板,没有直接提供\verb|nuaa.png|和\verb|nuaa.bst|文件,可以使用强制编译的方法生成文件,但是缺少左上角南航字样,参考文献格式也不符合标准。除此之外,旧模板使用了已经被放弃使用的CJK宏包,因此在编译\verb+\Unicode{}+命令时会出错,代码的阅读性和维护性也不如现在的ctex和xeCJK。由于上述原因,许多初次使用\LaTeX 和使用经验不多的同学,在一开始就放弃了使用旧版模板进行毕业设计的书写及排版。

基于南航无可用\LaTeX 学位模板可用的现状,\nuaathesis~Group基于旧\oldnuaathesis 模板、现东南大学的\seuthesix 模板、上海交通大学的SJTU Thesis模板和重庆大学 CQUThesis 模板,进行了二次开发,基本实现了本硕博学位论文的模板。

现在\nuaathesis 模板的代码托管在Github\footnote{\url{https://github.com/nuaatug/nuaathesis}}上,如有修改建议或者其他要求欢迎在Github上开issue或提pull request,\nuaathesis~Group会尽快回复,并酌情处理您的需求。

本模板当前版本基于 Windows 平台 \TeX Live 2017 开发,并用 CI 在其他平台下测试,但尚未在Windows平台使用Mik\TeX 进行测试。
% 本模板基于Windows~10平台开发,使用MiKTeX v2.9发行版,所使用的宏包均跟进到最新版本。Linux平台由张一白使用\TeX Live测试,macOS平台由王成欣进行了测试,目前尚未出现任何问题。本模板尚未在Windows平台使用\CTeX / \TeX Live进行测试,
如出现问题,请自行Google、Bing、Baidu搜索解决方法。学会使用搜索引擎、熟练阅读外文是一个学生最基本的能力,更是一个\LaTeX 使用者得以立足和前进的根本。

\nuaathesis~Group非常欢迎其他南航的\LaTeX 使用者加入到本模板的开发与维护当中来,不断完善模板,为南航广大学子造福!

\section{模板使用}
\subsection{准备工作}
\begin{itemize}
  \item \TeX 发行版:Windows 系统推荐使用MiK\TeX 和\TeX Live这两种发行版,前者占用空间小,只在有宏包缺失情况下才进行下载,后者占用空间大,但基本无需担心宏包缺失。Linux系统(Arch系除外)推荐手动安装\TeX Live发行版,官方源中的 \TeX Live 版本跟进较慢。macOS可以参考SJTU Thesis中的介绍。
  \item \TeX 知识:本说明文档提供\TeX 使用的例子,但不能解决所有的问题,因此使用前请自行学习\TeX~\&~\LaTeX 相关知识。
\end{itemize}

\subsection{模板编译}

我们推荐使用根目录下的\verb|.bat|、\verb|.sh|脚本进行编译。若需要手动编译,\textbf{切记使用\XeLaTeX 进行编译。}使用\XeLaTeX 和\hologo{BibTeX}在文档中加入参考文献的流程可参考如下命令:

\iffalse
\begin{lstlisting}[basicstyle=\small\ttfamily, caption=手动逐次编译, numbers=none]
xelatex -no-pdf .tex文件名
bibtex .tex文件名
xelatex .tex文件名
xelatex .tex文件名
\end{lstlisting}
\else
\begin{verbatim}
xelatex -no-pdf .tex文件名
bibtex .tex文件名
xelatex .tex文件名
xelatex .tex文件名
\end{verbatim}

注:因为已知bug,请在验证自己的环境后再使用 \texttt{lstlisting} 包,参见\url{https://github.com/CTeX-org/ctex-kit/issues/297}。

\fi

使用\XeLaTeX 引擎编译可以直接通过各\LaTeX 编辑器实现,如:TeXworks,TeXmaker,TeXStudio,Emacs+插件,Atom+插件等等。biber命令需使用Windows的cmd/Power Shell、Linux和macOS下的bash实现。

提示:Windows命令行用户,可以使用\verb|chcp 65001| 命令,将字符编码切换成 UTF-8,避免输出乱码。Windows 7系统及更高版本可使用 PowerShell 避免输出乱码。

使用biber需注意:\textbf{\texttt{.bib} 文件内的文件记录必须在 \texttt{.tex} 文件中被引用,不引用的记录不要因为懒而不去除,否则将编译失败。}

目录内容需要编译两次才能正常显示,原因推断为早期的电脑内存不够,所以将目录的生成分成了两步来进行。

\subsection{模板文件结构}
\subsubsection{必需文件}
\begin{itemize}[noitemsep,topsep=0pt,parsep=0pt,partopsep=0pt]
  \item \verb|.tex|文件:\TeX 主文件,chapter文件夹下有各个章节的文件,强烈建议将文章模块化,方便调试与版本管理;
  \item \verb|.bst|、\verb|.dtx|、\verb|.ins|文件:模板定义文件,不可删除;
  \item \verb|.bib|文件:参考文献数据库文件;
  \item \verb|.bat|、\verb|.sh|文件:批处理脚本,Windows系统下双击\verb|\bat|使用,Linux \& macOS 可在终端中使用\verb|.sh|;
  \item figure文件夹:存放要插入的图片,请不要删改已有的文件,这些文件是模板的一部分
\end{itemize}

\subsubsection{高级用户文件}
\begin{itemize}[noitemsep,topsep=0pt,parsep=0pt,partopsep=0pt]
  \item \verb|.ci|、\verb|.circleci|文件夹:持续集成配置文件夹,有需求的同学可直接使用,无需求的同学可以直接删除;
  \item \verb|.appveyor.yml|、\verb|.travis.yml|文件:持续集成配置文件,有需求的同学可直接使用,无需求的同学可以直接删除;
  \item \verb|.gitignore|文件:Git版本管理系统配置文件,有需求的同学可直接使用,无需求的同学可以直接删除
\end{itemize}