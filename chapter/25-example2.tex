\chapter{一些来自 CQUThesis 的使用样例}

\section{字体命令}\label{txt:FreqCmd}
{\kaishu 玲珑骰子安红豆,入骨相思知不知。\hfill ——温庭筠}

{\fangsong 愿得一心人,白头不相离。\hfill ——卓文君}

{\ifcsname youyuan\endcsname\youyuan\else[无 \texttt{youyuan} 字体。]\fi 去年今日此门中,人面桃花相映红。\hfill ——崔护}

{\heiti 入我相思门,知我相思苦。\hfill ——李白}

{\ifcsname lishu\endcsname\lishu\else[无 \texttt{lishu} 字体。]\fi 此情可待成追忆?只是当时已惘然。\hfill ——李商隐}

{\songti 雨打梨花深闭门,忘了青春,误了青春。\hfill ——唐寅}

使用\texttt{textbf}和\texttt{textit}以及\texttt{underline}的效果分别如下:

这句话的\textbf{文字}分别\textit{使用}了三种命令来\underline{处理}。

The \textbf{words} in this sentences are \textit{processed} with three different \underline{cmd}.

(注:粗宋体可能会被替换成黑体,参见 \autoref{txt:issue:boldsun})


\section{表格样本}

\subsection{基本表格}
\label{sec:basictable}

我们经常会在表格下方标注数据来源,或者对表格里面的条目进行解释。前面的脚注是一种不错的方法,如果不喜欢脚注,可以在表格后面写注释,比如\autoref{tab:tabexamp1}。
\begin{table}[htbp]
    \centering
    \bicaption{复杂表格示例}{A more structured table}
    \label{tab:tabexamp1}
    \begin{minipage}[t]{0.8\textwidth}
    \begin{tabularx}{\linewidth}{|l|X|X|X|X|}
        \hline
        \multirow{2}*{\diagbox[width=5em]{x}{y}} & \multicolumn{2}{c|}{First Half} & \multicolumn{2}{c|}{Second Half}\\\cline{2-5}
        & 1st Qtr &2nd Qtr&3rd Qtr&4th Qtr \\ \hline
        East$^{*}$ &   20.4&   27.4&   90&     20.4 \\
        West$^{**}$ &   30.6 &   38.6 &   34.6 &  31.6 \\ \hline
    \end{tabularx}\\[2pt]
    \footnotesize
    *:东部\\
    **:西部
    \end{minipage}
\end{table}

此外,\autoref{tab:tabexamp1} 同时还演示了另外两个功能:1)通过 \texttt{tabularx} 的\texttt{|X|} 扩展实现表格自动放大;2)通过命令 \texttt{diagbox} 在表头部分插入反斜线。

\autoref{tab:performance} 是一个很长的表格。

\begin{longtable}[c]{c*{6}{r}}
    \bicaption[实验数据]{实验数据,这个题注是双语的,而且十分的长,注意这在索引中的处理方式}[Data in experiment]{Data in experiment, and this is a really long long long long long long long long long long long long long long text.}\label{tab:performance}\\
    \toprule
    测试程序 & \multicolumn{1}{c}{正常运行} & \multicolumn{1}{c}{同步} & \multicolumn{1}{c}{检查点} & \multicolumn{1}{c}{卷回恢复}
    & \multicolumn{1}{c}{进程迁移} & \multicolumn{1}{c}{检查点} \\
    & \multicolumn{1}{c}{时间 (s)}& \multicolumn{1}{c}{时间 (s)}&
    \multicolumn{1}{c}{时间 (s)}& \multicolumn{1}{c}{时间 (s)}& \multicolumn{1}{c}{
        时间 (s)}&  文件 (KB) \\\midrule
    \endfirsthead
    \multicolumn{7}{c}{续表~\thetable\hskip1em 实验数据}\\
    \toprule
    测试程序 & \multicolumn{1}{c}{正常运行} & \multicolumn{1}{c}{同步} & \multicolumn{1}{c}{检查点} & \multicolumn{1}{c}{卷回恢复}
    & \multicolumn{1}{c}{进程迁移} & \multicolumn{1}{c}{检查点} \\
    & \multicolumn{1}{c}{时间 (s)}& \multicolumn{1}{c}{时间 (s)}&
    \multicolumn{1}{c}{时间 (s)}& \multicolumn{1}{c}{时间 (s)}& \multicolumn{1}{c}{
        时间 (s)}&  文件 (KB) \\\midrule
    \endhead
    \hline
    \multicolumn{7}{r}{续下页}
    \endfoot
    \endlastfoot
    CG.A.2 & 23.05 & 0.002 & 0.116 & 0.035 & 0.589 & 32491 \\
    CG.A.4 & 15.06 & 0.003 & 0.067 & 0.021 & 0.351 & 18211 \\
    CG.A.8 & 13.38 & 0.004 & 0.072 & 0.023 & 0.210 & 9890 \\
    CG.B.2 & 867.45 & 0.002 & 0.864 & 0.232 & 3.256 & 228562 \\
    CG.B.4 & 501.61 & 0.003 & 0.438 & 0.136 & 2.075 & 123862 \\
    CG.B.8 & 384.65 & 0.004 & 0.457 & 0.108 & 1.235 & 63777 \\
    MG.A.2 & 112.27 & 0.002 & 0.846 & 0.237 & 3.930 & 236473 \\
    MG.A.4 & 59.84 & 0.003 & 0.442 & 0.128 & 2.070 & 123875 \\
    MG.A.8 & 31.38 & 0.003 & 0.476 & 0.114 & 1.041 & 60627 \\
    MG.B.2 & 526.28 & 0.002 & 0.821 & 0.238 & 4.176 & 236635 \\
    MG.B.4 & 280.11 & 0.003 & 0.432 & 0.130 & 1.706 & 123793 \\
    MG.B.8 & 148.29 & 0.003 & 0.442 & 0.116 & 0.893 & 60600 \\
    LU.A.2 & 2116.54 & 0.002 & 0.110 & 0.030 & 0.532 & 28754 \\
    LU.A.4 & 1102.50 & 0.002 & 0.069 & 0.017 & 0.255 & 14915 \\
    LU.A.8 & 574.47 & 0.003 & 0.067 & 0.016 & 0.192 & 8655 \\
    LU.B.2 & 9712.87 & 0.002 & 0.357 & 0.104 & 1.734 & 101975 \\
    LU.B.4 & 4757.80 & 0.003 & 0.190 & 0.056 & 0.808 & 53522 \\
    LU.B.8 & 2444.05 & 0.004 & 0.222 & 0.057 & 0.548 & 30134 \\
    CG.B.2 & 867.45 & 0.002 & 0.864 & 0.232 & 3.256 & 228562 \\
    CG.B.4 & 501.61 & 0.003 & 0.438 & 0.136 & 2.075 & 123862 \\
    CG.B.8 & 384.65 & 0.004 & 0.457 & 0.108 & 1.235 & 63777 \\
    MG.A.2 & 112.27 & 0.002 & 0.846 & 0.237 & 3.930 & 236473 \\
    MG.A.4 & 59.84 & 0.003 & 0.442 & 0.128 & 2.070 & 123875 \\
    MG.A.8 & 31.38 & 0.003 & 0.476 & 0.114 & 1.041 & 60627 \\
    MG.B.2 & 526.28 & 0.002 & 0.821 & 0.238 & 4.176 & 236635 \\
    MG.B.4 & 280.11 & 0.003 & 0.432 & 0.130 & 1.706 & 123793 \\
    MG.B.8 & 148.29 & 0.003 & 0.442 & 0.116 & 0.893 & 60600 \\
    LU.A.2 & 2116.54 & 0.002 & 0.110 & 0.030 & 0.532 & 28754 \\
    LU.A.4 & 1102.50 & 0.002 & 0.069 & 0.017 & 0.255 & 14915 \\
    LU.A.8 & 574.47 & 0.003 & 0.067 & 0.016 & 0.192 & 8655 \\
    LU.B.2 & 9712.87 & 0.002 & 0.357 & 0.104 & 1.734 & 101975 \\
    LU.B.4 & 4757.80 & 0.003 & 0.190 & 0.056 & 0.808 & 53522 \\
    LU.B.8 & 2444.05 & 0.004 & 0.222 & 0.057 & 0.548 & 30134 \\
    EP.A.2 & 123.81 & 0.002 & 0.010 & 0.003 & 0.074 & 1834 \\
    EP.A.4 & 61.92 & 0.003 & 0.011 & 0.004 & 0.073 & 1743 \\
    EP.A.8 & 31.06 & 0.004 & 0.017 & 0.005 & 0.073 & 1661 \\
    EP.B.2 & 495.49 & 0.001 & 0.009 & 0.003 & 0.196 & 2011 \\
    EP.B.4 & 247.69 & 0.002 & 0.012 & 0.004 & 0.122 & 1663 \\
    EP.B.8 & 126.74 & 0.003 & 0.017 & 0.005 & 0.083 & 1656 \\
    \bottomrule
\end{longtable}

\section{定理环境}
\label{sec:theorem}

给大家演示一下各种和证明有关的环境:

\begin{assumption}
    假设以下数学方程成立:
    \begin{eqnarray}
    \label{eq:eqnxmp}
    c & = & a^2 - b^2\\
    & = & (a+b)(a-b)
    \end{eqnarray}
\end{assumption}

\begin{assumption}
    依然假设以下数学方程成立,注意整个系统是自动编号的:
    \begin{eqnarray}
    \label{eq:eqnxmp2}
    c & = & a^2 - b^2\\
    & = & (a+b)(a-b)
    \end{eqnarray}
\end{assumption}

\begin{cor}
    四川话配音的《猫和老鼠》是世界上最好看最好听最有趣的动画片。
    \begin{alignat}{3}
    V_i & =v_i - q_i v_j, & \qquad X_i & = x_i - q_i x_j,
    & \qquad U_i & = u_i,
    \qquad \text{for $i\ne j$;}\label{eq:B}\\
    V_j & = v_j, & \qquad X_j & = x_j,
    & \qquad U_j & u_j + \sum_{i\ne j} q_i u_i.
    \end{alignat}
\end{cor}

迢迢牵牛星,皎皎河汉女。
纤纤擢素手,札札弄机杼。
终日不成章,泣涕零如雨。
河汉清且浅,相去复几许。
盈盈一水间,脉脉不得语。

\begin{exmp}
    大家来看\autoref{ktc}。
    \begin{equation}
    \label{ktc}
    \left\{\begin{array}{l}
    \nabla f({\mbox{\boldmath $x$}}^*)-\sum\limits_{j=1}^p\lambda_j\nabla g_j({\mbox{\boldmath $x$}}^*)=0\\[0.3cm]
    \lambda_jg_j({\mbox{\boldmath $x$}}^*)=0,\quad j=1,2,\cdots,p\\[0.2cm]
    \lambda_j\ge 0,\quad j=1,2,\cdots,p.
    \end{array}\right.
    \end{equation}
\end{exmp}


\section{参考文献}
\label{sec:bib}
重庆大学的要求是参考文献以上标的形式标注于论述之后,就像这样:

研究表明\cite{r1},早睡早起有益身体健康。

如果想同时引用多个文献\cite{r2,r3,r4,r6},只需要在\verb|cite{}|中用逗号分开\textsf{citeKey}就好。
CQUThesis 同时提供正文模式的参考文献引用功能\texttt{inlinecite},适用于以下情况:
文献\inlinecite{r6}表明,文献\inlinecite{r7,r8,r9}所述的情况是有理论依据的。
