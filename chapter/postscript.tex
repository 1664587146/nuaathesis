\chapter{v0.9a后记——Old Jack 的吐槽}

\verb!\begin{轻松+愉快}!

Old Jack 他有点累......

Old Jack 两年前就开始关注南航毕设的\LaTeX 模板了,但是两年了还没有任何有实际意义的新动作,所以Old Jack 就亲自操刀制作了新的一版。虽然很多代码都是从其他模板中直接摘抄过来的,但是这也是\TeX 最普遍、最快捷的学习\&开发方法。一开始 Old Jack 也想造轮子,但是轮子真的不好造。

在制作过程中遇到了几个关键性的问题:
\begin{itemize}
  \item 前文提到的三种粗体
  \item nuaa.png源文件和页眉制作
  \item 英文字母、章节标题莫名其妙的加粗
  \item 脚注相对页脚线的位置
\end{itemize}

第一个问题 Old Jack 曾经用\TeX 中伪粗体(FakeBold)的方法实现过,但是效果并不好,而且当时受到最后一个问题的强烈影响,不得不使用其他字体来解决这个问题。

第二个问题 Old Jack 开始是使用官方模板中的图片,但是分辨率太低,效果很差。于是 Old Jack Google以图搜图找到了现在的这个文件的源文件,经过了一系列不可描述的操作后得到了现在的 nuaa.png 。页眉的制作也让 Old Jack 很头疼,论文要求论文到顶端和底端的距离分别为2.5cm和2.0cm,Old Jack 很naive的就给geometry设置了这个数值,但是效果和官方模板差了很多,于是 Old Jack 只好一点一点地调试,达到了近似官方模板的效果。页脚和官方模板有细微的区别,Old Jack 认为这无伤大雅,是要罗马数字和阿拉伯数字编号正确应该就可以了。

第三个问题是一个非常奇怪的问题。使用伪粗体时所有标题全都加粗了,非常难看,经过了代码重构和不停地调试解决了这个问题。在模板完成99\% 后发现最后致谢中的英文字体全都加粗了, Old Jack 几次审视代码和调试都没有解决。偶然间,Old Jack 将全部主要文件全部提取出来,放入另一个文件夹,然后重新编译就解决了这个问题!当然后来发现代码中确实有一个地方有小问题\textbf{可能}会影响,但是这不是上一次出错的原因。Old Jack 对于各位使用模板的南航学子以及其他可能会参考此模板的\TeX 爱好者提了一个建议:\textbf{任何语言,任何代码出现莫名其妙的问题时,换一个文件夹,改一下名字,重新跑一下,可能会得到意想不到的结果。}当然这不是万能的解决方法。

第四个问题就如第一章中脚注和页脚线的情况,感觉两条线很别扭。 Old Jack 犹豫了很久,最后没有采用将脚注放在页脚线下的方案,因为 Old Jack 觉得还是两条线的方案好看。对于想要将脚注放在页脚线下方的同学,可以在主文件中取消注释那段代码,来实现所需要的效果。

Old Jack 他完成了模板的再制作,但是他没有心气再写出一篇能够指导大家使用\LaTeX 的文档了(好吧,Old Jack 他承认懒是一部分因素),望大家谅解 Old Jack 。

\verb!\end{轻松+愉快}!
