% 本文件是示例论文的一部分
% 论文的主文件位于上级目录的 `bachelor.tex` 或 `master.tex`

\chapter{快速上手}

\section{欢迎}

欢迎使用 \nuaathesis,本文档将介绍如何利用 \nuaathesis 模板进行学位论文写作,
我们假设读者有 \LaTeX 英文写作经验,并会使用搜索引擎解决常见问题。

本模板的源代码托管在 \url{https://github.com/nuaatug/nuaathesis},
欢迎来提 issue/PR。

\section{\LaTeX 环境准备}

由于本模板使用了大量宏包,因此对 \LaTeX 环境有不少要求。
推荐使用以下打 \ding{51} 的 \LaTeX 发行版:
\begin{itemize}
\item[\ding{51}]\TeX~Live 请安装以下 collection:langchinese, latexextra, science, pictures, fontsextra;\\
如果觉得安装体积太大的话,可以看 \texttt{.ci/texlive.pkgs} 列出的所需宏包;
\item[\ding{51}]MiK\TeX 请祈祷国内的镜像服务器不抽风;如果它抽风了,建议隔天再试; \\
因为 MiK\TeX{} 能自动下载安装宏包,非常推荐 Windows 用户使用。
\item[\ding{53}]CTeX (\url{http://www.ctex.org/}) 不推荐,可能宏包缺失、版本过旧导致无法编译。
\end{itemize}

\section{编译模板和文档}

只有在找不到 \verb|nuaathesis.cls| 文件的时候,才需要执行本步骤。

进入模板的根目录,运行 \verb|build.bat|(Windows) 或 \verb|build.sh|(其他系统),
它会生成模板 \verb|nuaathesis.cls| 以及对应的文档 \verb|nuaathesis.pdf|。

\section{使用模板}

论文写作时,请确认\textbf{论文的目录}下有以下文件:
\begin{itemize}
  \item \verb|nuaathesis.cls| 文档模板;
  \item \verb|nuaathesis.bst| 参考文献格式(如果使用 biber 来生成参考文献的话);
  \item \verb|logo/| 文件夹,内含一些图标;
\end{itemize}

如果论文目录下没有这些文件的话,请从本模板根目录复制一份。

\section{开始写作}

最方便的开始方法,莫过于修改现有的文稿。因此推荐直接修改本文档:
\begin{itemize}
  \item \verb|bachelor.tex| 或 \verb|master.tex| 主文件,定义了文档包含的内容。建议删除并只保留其中一个主文件;
  \item \verb|global.tex| 里面定义文档的信息,导入一些宏包,并设置全局使用的宏;
  \item \verb|content/| 文件夹,按章节拆分的文档内容,这里;
  \item \verb|ref/| 文件夹,内含参考文献数据库;
\end{itemize}

修改完成后,使用 \verb|latexmk -xelatex bachelor| 进行编译。
如果需要使用图形界面的编辑器的话,请继续阅读本节内容。

\subsection{使用 TeXstudio}
\begin{enumerate}
\item 打开主文件 \verb|bachelor.tex| 或 \verb|master.tex|;
\item 菜单 Options > Configure TeXstudio 对话框;
\item 左侧选择 Build,右侧将 Default Compiler 修改为 Latexmk;
\item 确认即可。
\end{enumerate}

\subsection{使用 vscode}
\begin{enumerate}
\item 打开论文目录;
\item 安装 LaTeX Workshop 插件;
\item 打开论文的主文件 \verb|bachelor.tex| 或 \verb|master.tex|,删除没有用到的主文件;
\item 使用 LaTeX Workshop 插件提供的编译命令编译文档。
\end{enumerate}

\section{打印论文}

如果论文需要双面打印的话,请务必修改文档类选项,编译双面打印用的 PDF 文件。
具体地说,在主文件的头部,去除 \texttt{openany, oneside},改成 \texttt{openright, blankleft, twoside}。
