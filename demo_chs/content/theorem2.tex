\chapter{定理环境·下}

本章演示使用 \cs{nuaatheoremchap} 定义的定理环境,注意它们的数字编号是可以重复的。

\section{演示}

\begin{nonsense}
\label{non:dora}
哆啦A梦写的论文被拒稿的可能性很高
\footnote{出处:\url{https://www.math.kyoto-u.ac.jp/~arai/latex/presen2.pdf} 的最后一页}。
\end{nonsense}

\begin{exercise}
\label{ex:oneplus}
证明$1+1 = 2$。
\end{exercise}

\begin{nonsense}[右边的胡诌是真的]
“练习”与“胡诌”定理环境的编号是相互独立的,它们的数字编号允许重复,
如“\autoref{non:dora}”和“\autoref{ex:oneplus}”。
\end{nonsense}

\begin{exercise}
按照本文所演示的方法,利用 \cs{nuaatheorem(g|chap|chapu)} 来定义您的论文中所需要的定理环境。
\end{exercise}

\begin{lines}
\label{s2}
例句2
\end{lines}

\autoref{s2} 没有章节编号,它是全局编号的,它可以用在外国语学院论文中来枚举例句。
